\documentclass[10pt]{beamer}

\usetheme{metropolis}

\usepackage{graphicx}
\usepackage{xspace}

\def\numpost#1{$^{#1}\;$}
\def\numpre#1{$^{#1}\!$}

\def\twopanels#1#2{\only<#1>{%
    \includegraphics[height=.33\linewidth]{figs/images#2.png}
    \includegraphics[height=.33\linewidth]{figs/correl#2.png}}}

\def\imwid#1#2{\includegraphics[width=#1\linewidth]{fig/#2}}
\def\imht#1#2{\includegraphics[width=#1\vsize]{fig/#2}}

\title{Generative AI for Image Reconstruction:\\
  a First Attempt}

\date{}

\author{Y. van~der~Burg,\numpost1 N. Rai,\numpost2
  S. Basak,\numpost2, S. Sarangi,\numpost3 P. Saha,\numpost1 }

\institute{\numpre1 Uni Zurich CH, \numpre2 IISER-TVM India, 
  \numpre3 CUTM India}

\begin{document}

\maketitle

\begin{frame}{Can we adapt this?}
\centering
\imwid{0.8}{Isola-fig}
\vskip5pt 
\texttt{https://phillipi.github.io/pix2pix/}    
\end{frame}

\begin{frame}{Gravity Darkening}
\centering
\hbox to \hsize{\hss\imwid{1.2}{regulus_che11}\hss}
\vskip5pt 
Che et al (2009) using CHARA
\end{frame}

\begin{frame}{Make a training set}
\centering
\only<1>{\imwid{.4}{ellipse4678}\quad\imwid{.4}{ellipse5972}}%
\only<2>{\imwid{.4}{ellipse6018}\quad\imwid{.4}{ellipse6098}}%
\only<3>{\imwid{.4}{ellipse6018}\quad\imwid{.4}{ft_log}}%
\only<4>{\imwid{.4}{ellipse6018}\quad\imwid{.4}{ft_log_base}}
\vskip4pt
\visible<4>{The training set has 60\thinspace 000 of these.}
\end{frame}

\begin{frame}{A cGAN}

\begin{itemize}
\item One network (the generator) produces images from
  sparse II data.
\item A second network (the discriminator) separates good and bad
  images.
\item These are trained alternately.
\end{itemize}   

\centering
\imwid{0.45}{Plot_N_telescopes_gen_total_loss}%
\imwid{0.45}{Plot_N_telescopes_disc_loss}%
\end{frame}

\begin{frame}{Results}
\only<1>{\hbox to \hsize{\hss\imwid{1.2}{image_0}\hss}%
         \hbox to \hsize{\hss\imwid{1.2}{image_16}\hss}}
\only<2>{\hbox to \hsize{\hss\imwid{1.2}{image_35}\hss}%
         \hbox to \hsize{\hss\imwid{1.2}{image_38}\hss}}
\only<3>{\hbox to \hsize{\hss\imwid{1.2}{image_42}\hss}%
         \hbox to \hsize{\hss\imwid{1.2}{image_47}\hss}}
\end{frame}


\begin{frame}{Recovery of Multipoles}
\centering
\imht{0.7}{mom0}
\vfil
Monopole is well recovered.
\end{frame}

\begin{frame}{Recovery of Multipoles}
\centering
\only<1>{\imht{0.7}{mom3}}%
\only<2>{\imht{0.7}{mom4}}%
\only<3>{\imht{0.7}{mom5}}
\vfil
Second moment is also well recovered.
\end{frame}

\begin{frame}{Recovery of Multipoles}
\centering
\only<1>{\imht{0.7}{mom6}}%
\only<2>{\imht{0.7}{mom7}}%
\only<3>{\imht{0.7}{mom8}}%
\only<4>{\imht{0.7}{mom9}}
\vfil
Third moment is less good.
\end{frame}

\begin{frame}{Summary}
\begin{itemize}
\item<1->{Adapting an off-the-shelf code gives encouraging results for
  reconstructing gravity-darkening using ${}^4C_2$ baselines.}
\item<2->{Interpretion of loss functions desirable.}
\item<3>{Next step: simulations of interacting binaries as training set?}
\end{itemize}
\end{frame}

\end{document}
