\section{Introduction}

Intensity Interferometry (II) emerged in the 1970s as a groundbreaking technique for measuring the diameters, orbits, and limb-darkening coefficients of 32 stars in single/binary/multiple star systems \cite{brown1974intensity}. Despite these significant achievements, the method did not gain widespread adoption due to the then-required highly sensitive photon detectors and sophisticated data analysis techniques. However, advancements in computational methods and modern electronics, which now offer time resolution of detectors in the nanoseconds (ns) range, have renewed II as a feasible technique in comparison to the limitation of amplitude interferometry for resolving astrophysical objects at visible wavelengths. Current simulations with ns resolution of photon detectors show that II is effective in achieving high precision measurement of parameters for stellar objects \citep{10.1093/mnras/stab2391, 10.1093/mnras/stac2433}. However, despite these technological advancements, II faces a significant limitation in terms of phase loss and the absence of complete reconstruction of the brightness distribution of stellar sources.

Several theoretical and computational approaches to phase reconstruction with II have been proposed. \cite{gamo1963triple} introduced the concept of triple-intensity correlation, which \cite{goldberger1963use} further applied in an experiment to observe scattered particles in microscopic systems. Sato conducted experiments to measure the diameter and phase of asymmetrical objects, suggesting that triple correlation could extend II to image stellar bodies \citep{sato1978imaging, sato1979computer, sato1981adaptive}. Despite its potential, the sensitivity of the signal to noise poses a significant challenge for this approach.

Later \cite{holmes2010two} proposed an alternative method, utilizing the Cauchy-Riemann relations to reconstruct 1-D images and extending this to 2-D images across a range of signal-to-noise (SNR) values. This algorithm was applied to simulated data of stellar objects using II, considering existing and forthcoming Imaging Cherenkov Telescope Arrays (ICTAs) with a large number of telescopes \citep{nunez2010stellar, nunez2012high, nunez2012imaging}. Despite these theoretical advances, practical application to actual observational data of stellar objects with II has yet to be achieved. The introduction of a novel image reconstruction technique with II will advance the field of optical astronomy.

Currently, Artificial Intelligence (AI) is applied to high-volume data to reduce human computing resources and time effort. New analysis techniques using Machine Learning (ML) applications on observed data are playing at the front to reveal the astronomical mystery. In radio morphologies, the supervised and unsupervised algorithm is used to distinguish the radio components from the infrared source \citep{wu2019radio, galvin2020cataloguing}. Recently, the image reconstruction of the M87 black hole using Generative Adversarial Networks (GANs) has been performed \citep{10.1093/mnras/stad3797}. There is also a need to introduce the application of ML to II data, which will aim to overcome the limitations of image reconstruction and provide high-resolution optical astronomy. 

In this paper, we propose an alternative approach for image reconstruction of stellar objects with II using conditional GAN (cGAN) \citep{goodfellow2014generative}. We consider four ICTAs and perform a one-night simulation for a fast-rotating star. The predicted image by a trained GAN shows promising results for reconstructing the image's shape and size. Image moments evaluate the predicted image and demonstrate that the size and shape of the fast-rotating star are reconstructed. However, the brightness distribution recovers only up to the third-order moment. We expect even greater precision in image reconstruction by including more telescopes, especially in recovering higher-order image moments. It holds great promise, pushing the boundaries of high-resolution optical astronomy and opening up new possibilities for studying stellar objects with remarkable detail.

This paper is organized as follows: The first section discusses Intensity Interferometry following its signal and noise for fast-rotating stars along the Earth's rotation. The second section introduces the GAN formulation and its structure. The third section details the parameter selection for training the GAN to reconstruct the image. After that, the fourth section presents the results of trained GAN visually using image moments. At the end, there is a conclusion of the entire result.