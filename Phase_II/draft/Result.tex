\section{The Reconstructed Image with GAN}

In this section, we will first discuss phase retrieval using hyperparameters, as mentioned earlier, followed by an analysis where multiple sources are trained simultaneously. The best performance for image reconstruction has been observed with a learning rate of $2 \cdot 10^{-4}$, a kernel size of 5x5, and equal noise percentages in original and generated images. The batch size selected is 1, and the Discriminator-Generator receives equal training.

\subsection{The Predicted Image with Trained GAN}
The success of the GAN in training the model for Intensity Interferometry to reconstruct images of fast-rotating stars is demonstrated in Fig.~\ref{fig:GAN}. The GAN was trained on training datasets for 60,000 steps. After training, the GAN was tested on different validation datasets, producing predicted images of a fast-rotating star. Fig.~\ref{fig:GAN} presents a set of four combined images demonstrating the GAN's performance in reconstructing the shape, size, and brightness distribution of the fast-rotating star using II. The left panel shows the signals covered by six baselines, which serve as the input for the Generator to train the GAN. The first middle panel displays the real image, or ground truth, which the Discriminator uses to differentiate between real and generated images from the Generator. During training, the GAN aims to mimic these ground truth images. The second middle panel depicts the reconstructed image, or predicted image, produced by the trained GAN. This panel illustrates the success of the GAN in image reconstruction using II. The right panel shows the difference between the ground truth and the predicted image. This difference should be minimized for high-precision image reconstruction with the application of GAN on II. The predicted images of Fig.~\ref{fig:GAN} showed positive results. It accurately provides visual information about the source's size, shape, and brightness distribution over the surface. It is achieved here using only six baselines. However, the results can be further improved by increasing the number of telescopes to cover maximum (u, v) planes, making the existing and upcoming Cherenkov Telescope Array Observatory (CTAO) an ideal candidate. 

\subsection{Evaluation of GAN using Moments}
The reconstructed images are good enough visually to claim the success of GAN in reconstructing the image using II. However, there is a need for statistical evaluation of this predicted image to guarantee success. So, we use image moments as a statistical method. Image moments are statistical properties that provide information about the reconstructed shape, size, and intensity distribution of the objects in the image. These moments help to quantify key features like the position, orientation, and distribution of brightness in the image, allowing for an objective evaluation of how well the predicted image corresponds to the actual object. We will analyze the consistency and accuracy of the GAN-generated image by comparing its moments to those of the ground truth. It provides a reliable framework for evaluating reconstruction quality, as image moments highlight the subtle differences in geometric and intensity properties that might not be visually apparent. 

The raw moment $M_{ij}$ of an image I(x, y) is defined as \citep{hu1962visual}
\begin{equation}
	M_{ij} = \sum_{x} \sum_{y} x^i y^j I(x, y).
	\label{eqn:Mom}
\end{equation}

The zeroth order raw moment called the monopole is the total intensity of an image. It sums up all the pixel values across the image, providing an overall intensity value. So, the study of monopole provides the total flux of fast-rotating stars here. According to eqn.~\ref{eqn:Mom}, the monopole of an image is calculated as 
\begin{equation}
	M_{00} = \sum_{x} \sum_{y} I(x, y).
\end{equation}
Fig.~\ref{fig:mom1} shows the monopole for 50 different reconstructed images. It shows the linear behavior for the monopole between the ground truth (the real image) along the x-axis and the predicted image (reconstructed image) along the y-axis, which is obvious for different shape-size sources. This result explains the similarity between the total intensity of both images. It ensures that the predicted image has the approximately correct total brightness (the flux) compared to the ground truth. However, monopole does not explain the position, shape, size, and brightness distribution of fast-rotating stars; there is a need for higher-order moments.

The center of mass for the fast-rotating star or any other stellar object is calculated using the centroid (x-centroid and y-centroid). It represents the spatial position of the image and is calculated using first-order raw moment and monopole. The formulation of centroid along the x and y directions is
\begin{equation}
	\begin{aligned}
		&m_x = \frac{\sum_{x} x I(x,y)}{\sum_{x} \sum_{y} I(x, y)} = \frac{M_{10}}{M_{00}} \\
		&m_y = \frac{\sum_{y} y I(x,y)}{\sum_{x} \sum_{y} I(x, y)} = \frac{M_{01}}{M_{00}}
	\end{aligned}  
\end{equation}
Fig.~\ref{fig:mom2} and Fig.~\ref{fig:mom3} show the comparison of the x-centroid and y-centroid for 50 predicted images with respect to ground truths, respectively. The clustering of centroids in a given scale range for all the results explains that the reconstructed image correctly represents the spatial location of the fast-rotating star compared with the ground truth.

Further, these calculated centroids will help to study the shape, size, and brightness distribution of fast-rotating stars in terms of higher-order image moments. For that, the central moment of an image is calculated according to
\begin{equation}
	\mu_{pq} = \frac{1}{M_{00}}\sum_{x} \sum_{y} (x - m_x)^p (y - m_y)^q I(x, y).
\end{equation}
The sum of $p$ and $q$ defines the order of the central moment.

The second order central moment ($\mu_{11}, \mu_{20}, \mu_{02}$) has been shown in Fig.~\ref{fig:struc}, which is used to study the structure of a fast-rotating star along the line of sight of observation (explain in upcoming subsection). All these three plots explain the linear relation of second-order moments again as for monopole and show the success of the application of GAN to reconstruct the image with II.

The information on brightness distribution is gathered using the skewness of the image by calculating third-order central moment ($\mu_{30}, \mu_{03}, \mu_{21}, \mu_{12}$) of images. Fig.~\ref{fig:moments} shows all third-order moments for the ground truth and reconstructed image. The skewness along the x and y axis ($\mu_{30}, \mu_{03}$) to test the GAN for II are acceptable, which can be seen in Fig.~\ref{fig:mom7} and Fig.~\ref{fig:mom8}, where a linear relation exists between ground truth and predicted image. However, the remaining higher moments $(\mu_{21}, \mu_{12})$ shown in Fig.~\ref{fig:mom9} and Fig.~\ref{fig:mom10} are not in good terms specially $\mu_{12}$. Further improvement is possible and needs to be studied.

\subsection{The Reconstructed Parameters for Object}
The centroids $(m_x, m_y)$ represent the center of the fast-rotating star and spatial location on the image only. However, the calculated second-order central moment defines the orientation, semi-major axis, and eccentricity with respect to the center of the source \citep{teague1980image}. These parameters based on the moments fully describe the two-dimensional ellipse that fits the image data. 

The orientation of a fast-rotating star along the line of sight is defined in terms of second-order central moments as
\begin{equation}
	\theta = \frac{1}{2}\arctan \big(\frac{2\mu_{11}}{\mu_{20} - \mu_{02}}\big).
	\label{eqn:orn}
\end{equation}
The semi-major and semi-minor axis of the stellar object will be calculated with the help of second-order central moment again and denoted as a and b here, respectively
\begin{equation}
	\begin{aligned}
		&a = 2\sqrt{mp + \delta} \\
		&b = 2\sqrt{mp - \delta}
	\end{aligned}
	\label{eqn:semi}
\end{equation}
where,
\begin{equation}
	mp = \frac{\mu_{20} + \mu_{02}}{2}
	\label{eqn:mp}
\end{equation}
and
\begin{equation}
	\delta = \frac{\sqrt{4\mu_{11}^2 + (\mu_{20} - \mu_{02})^2}}{2}.	
	\label{eqn:delta}
\end{equation}
With the help of the axis value, the eccentricity of the fast-rotating star is calculated as
\begin{equation}
	e = \sqrt{1 - a/b}.
	\label{eqn:eccen}
\end{equation}
Eqns.~\ref{eqn:orn}-\ref{eqn:eccen} depending on values explains the elliptical nature of the stellar object (here, fast-rotating star) and provides information on shape and size. However, the information about the brightness distribution is gathered using skewness, which can be understood using third and higher-order moments.