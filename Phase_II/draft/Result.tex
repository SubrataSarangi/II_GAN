\section{The Reconstructed Image with GAN}
In this section, we will first discuss phase retrieval using hyperparameters, as mentioned earlier, followed by an analysis where multiple sources are trained simultaneously. The best performance for image reconstruction has been observed with a learning rate of $2 \cdot 10^{-4}$, a kernel size of 5x5, and equal noise percentages in original and generated images. The batch size selected is 1, and the discriminator-generator receives equal training.
\subsection{The Predicted Image with Trained GAN}
The success of the GAN in training the model for Intensity Interferometry to reconstruct images of fast-rotating stars is demonstrated in Fig.~\ref{fig:GAN}. The GAN was trained on training datasets for 60,000 steps. After training, the GAN was tested on different validation datasets, producing predicted images of a fast-rotating star. Fig.~\ref{fig:GAN} presents a set of four combined images demonstrating the GAN's performance in reconstructing the shape, size, and brightness distribution of the fast-rotating star using II. The left panel shows the signals covered by six baselines, which serve as the input for the generator to train the GAN. The first, middle panel displays the real image, or ground truth, which the discriminator uses to differentiate between real and generated images from the generator. During training, the GAN aims to mimic this ground truth. The second middle panel depicts the reconstructed image, or predicted image, produced by the trained GAN. This panel illustrates the success of the GAN in image reconstruction using II. The right panel shows the difference between the ground truth and the predicted image. This difference should be minimized for high-precision image reconstruction with application of GAN on II. So, the predicted images of fig.~\ref{fig:GAN} showed positive results, accurately providing visual information about the source's size and shape, as well as the distributed brightness. This image reconstruction was achieved using only six baselines. However, the results can be further improved by increasing the number of telescopes to cover more (u, v) planes.
\subsection{Evaluation of GAN using moments}
After the success of GAN in reconstructing the image using II, there is a need for evaluation of this result. Here, we use the image moments for this work. Image moments are statistical properties that provide information about the shape, size, and intensity distribution of the objects in the image. So, we evaluate images generated by the GAN with the moments to ensure the accuracy and precision of the predicted image. 

The raw moment $M_{ij}$ of an image I(x, y) is defined as \citep{hu1962visual}
\begin{equation}
	M_{ij} = \sum_{x} \sum_{y} x^i y^j I(x, y).
	\label{eqn:Mom}
\end{equation}

The zeroth order raw moment, which is the total intensity of an image and is called the monopole. It sums up all the pixel values across the image, providing an overall intensity value. So, the study of monopole provides the total flux of fast-rotating stars here. According to eqn.~\ref{eqn:Mom}, the monopole of a image is calculated as 
\begin{equation}
	M_{00} = \sum_{x} \sum_{y} I(x, y).
\end{equation}
The left figure (fig.~\ref{fig:mom1}) in fig.~\ref{fig:cen} shows the monopole for 50 different reconstructed images. It shows the linear behavior for the monopole between the ground truth (the real image) along the x-axis and the predicted image (reconstructed image) along the y-axis, which is obvious for different shape-size sources. This result explains the similarity between the total intensity of both images. It ensures that the predicted image has the approximately correct total brightness (the flux) compared to the ground truth. However, monopole does not explain the shape, size, and brightness distribution of fast-rotating stars. So, we need higher-order moments.

The center of mass for the fast-rotating star or any other stellar object is calculated using the centroid (in x and y directions). These centroids are given in terms of first-order raw moment and monopole as
\begin{equation}
	\begin{aligned}
		&m_x = \frac{\sum_{x} x I(x,y)}{\sum_{x} \sum_{y} I(x, y)} = \frac{M_{10}}{M_{00}} \\
		&m_y = \frac{\sum_{y} y I(x,y)}{\sum_{x} \sum_{y} I(x, y)} = \frac{M_{01}}{M_{00}}
	\end{aligned}  
\end{equation}
Fig.~\ref{fig:mom2} and fig.~\ref{fig:mom3} show the comparison of centroid along the x and y axis for 50 predicted images with respect to ground truths. The clustering of centroids in a given scale range for all the results explains that the reconstructed image correctly represents the spatial location of the fast-rotating star.

Now, these centroids will help to study the shape, size, and brightness distribution of fast-rotating stars in terms of higher-order image moments. For that, the central moment of an image is calculated according to
\begin{equation}
	\mu_{pq} = \frac{1}{M_{00}}\sum_{x} \sum_{y} (x - m_x)^p (y - m_y)^q I(x, y)
\end{equation}

The second order moment ($\mu_{11}, \mu_{20}, \mu_{02}$) has been shown in fig.~\ref{fig:struc}, which is used to study the structure of a fast-rotating star along the line of sight of observation (explain in upcoming subsection). All these three plots explain the linear relation of second-order moments again as for monopole and show the success of the application of GAN to reconstruct the image with II.

The information on brightness distribution can be gathered using the skewness of the image, and for that, there is a need for third order ($\mu_{30}, \mu_{03}, \mu_{21}, \mu_{12}$) central moment of images. Fig.~\ref{fig:moments} shows all third-order moments for the ground truth and reconstructed image. The skewness along the x and y axis ($\mu_{30}, \mu_{03}$) to test the GAN for II are acceptable, which can be seen in fig.~\ref{fig:mom7} and fig.~\ref{fig:mom8} where a linear relation exists between ground truth and predicted image. However, the remaining higher moments $(\mu_{21}, \mu_{12})$ shown in fig.~\ref{fig:mom9} and fig.~\ref{fig:mom10} are not in good terms specially $\mu_{12}$.

\subsection{The reconstructed Parameters for object}
The centroids $(m-x, m_y)$ represents the center of the fast-rotating star. However, the calculated second-order central moment defines the orientation, semi-major axis, and eccentricity of the source \citep{teague1980image}. These parameters fully describe the ellipse that fits the image data based on the moments. 

The orientation along the line of sight is defined as
\begin{equation}
	\theta = \frac{1}{2}\arctan \big(\frac{2\mu_{11}}{\mu_{20} - \mu_{02}}\big).
	\label{eqn:orn}
\end{equation}
The semi-major and semi-minor axis will be calculated according to
\begin{equation}
	\begin{aligned}
		&a = 2\sqrt{mp + \delta} \\
		&b = 2\sqrt{mp - \delta}
	\end{aligned}
	\label{eqn:semi}
\end{equation}
where,
\begin{equation}
	mp = \frac{\mu_{20} + \mu_{02}}{2}
	\label{eqn:mp}
\end{equation}
and
\begin{equation}
	\delta = \frac{\sqrt{4\mu_{11}^2 + (\mu_{20} - \mu_{02})^2}}{2}.	
	\label{eqn:delta}
\end{equation}
So, the eccentricity of the fast-rotating star in terms of axis value is 
\begin{equation}
	e = \sqrt{1 - a/b}.
	\label{eqn:eccen}
\end{equation}
Fig.~\ref{fig:recons} shows all zeroth to third order moments only for one ground and predicted image. In this figure, there is a red curve for both ground truth and predicted image which is the representation of all second-order moments in terms of eqn.~\ref{eqn:orn} to eqn.~\ref{eqn:eccen} as orientation, axis value, and eccentricity. There is a cyan dot point as well, which represents the zeroth order moment for both images. There is also a third-order moment, which is expressed in terms of arrows in magenta and blue color to view the skewness along the x and y axes.
