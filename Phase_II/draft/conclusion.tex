\section{Conclusion}
The challenge of phase retrieval in Intensity Interferometry (II) has been effectively addressed through the application of machine learning techniques, particularly with Generative Adversarial Networks (GANs). Our study demonstrates that using a GAN on II data successfully recovers the size, shape, and brightness distribution of a fast-rotating star. The evaluation based on image moments, specifically the monopole, second, and third-order moments, supports the effectiveness of the GAN in achieving accurate image reconstruction from a one-night observation using six baselines.

A critical factor in the performance of the reconstruction process is the extent of Fourier plane coverage, which is determined by the number of available telescopes and the total observing time. The brightness distribution can likely be reconstructed with even higher precision with a full night of observation using four telescopes. Future work could explore different observatory layouts, such as the Southern Cherenkov Telescope Array (CTA) layout, to evaluate the impact on image reconstruction quality.

This study has not considered the detector efficiencies, which could influence the signal-to-noise ratio (SNR) in practical scenarios. Incorporating these efficiencies in future work will be important for providing a more accurate estimate of the SNR and the robustness of the GAN in real-world applications.

While conditional GANs have proven effective in this study, there are other methods for generating images from image data. Exploring and comparing alternative methods, as well as adapting the GAN architecture itself, could offer improvements. For instance, different loss functions could be implemented and compared to potentially enhance reconstruction quality. Further testing is required to refine the GAN and make it more robust and reliable. However, our findings suggest that machine learning is a promising approach for phase reconstruction in II.

In summary, while the current implementation shows strong potential, further exploration and refinement of machine learning models are needed to optimize their application in phase retrieval and image reconstruction in Intensity Interferometry.