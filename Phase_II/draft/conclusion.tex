\section{Conclusion}
Intensity Interferometry (II), a technique, is emerging after a long gap to overcome the challenges of Amplitude Interferometry with optical wavelength range. It is rejuvenated with support from advanced facilities at the Imaging Cherenkov Telescope Arrays (ICTAs). These telescopes are instrumental in capturing high-resolution images equipped with large aperture areas and sensitive photon detectors (capable of resolving signals within a nanosecond) \citep{dravins2013optical}. The Major Atmospheric Gamma Imaging Cherenkov Telescope (MAGIC), one of the CTA in the world, is now also being used for II, observing stellar objects with its 17-meter diameter mirror \citep{lorenz2004status}. The telescope has already demonstrated the potential of II by measuring and comparing the diameters of individual stars \citep{abe2024performance}. Other existing arrays (Very Energetic Radiation Imaging Telescope Array System (VERITAS) and High Energy Stereoscopic System (HESS)) are also making progress as II when gamma-ray observation stops \citep{kieda2021veritas, zmija2022optical}. Future observations, made possible by more advanced, sensitive instrumentation and upcoming CTAO, promise to push the boundaries of what we can observe with optical range. However, there is a drawback of II in terms of phase loss of signal, capturing only its magnitude via photon correlation. Here, the challenge of phase retrieval in Intensity Interferometry (II) has been effectively addressed through the application of machine learning techniques, particularly with conditional Generative Adversarial Networks (cGAN). Our study demonstrates that the application of cGAN on II data successfully recovers the size, shape, and brightness distribution of a fast-rotating star. The evaluation based on image moments, specifically the monopole, second, and third-order moments, supports the effectiveness of the cGAN in achieving accurate image reconstruction from a one-night simulation of II using six baselines. A critical factor in the performance of the reconstruction process is the extent of Fourier plane coverage, which is determined by the number of available telescopes and the total observing time. The brightness distribution can likely be reconstructed with even higher precision with a full night of observation using four telescopes. Future work could explore different observatory layouts to study the impact on image reconstruction quality, such as the Southern Cherenkov Telescope Array (CTA) with MAGIC.

While the results of this study demonstrate the significant potential of machine learning, particularly cGAN, for image reconstruction in II, several aspects need further refinement. (1) Detector efficiencies, which impact the signal-to-noise ratio (SNR) of real observational data, have yet to be incorporated. It will be crucial to address these challenges for more accurate SNR estimation. Incorporating these efficiencies in future work will be important for providing a more accurate estimate of the SNR. (2) Exploring and comparing different methods to generate images, which works better than cGAN to reconstruct stellar images with II. (3) Different loss functions can be used to check the reconstruction quality. So, there are further testing is required to refine the GAN and make it more robust and reliable. However, our findings suggest that machine learning is a promising approach for phase reconstruction in II.

In summary, though cGAN has shown promising results, further testing and improvements are necessary to make the model more robust and reliable in practical applications. Overall, this study highlights the potential of machine learning approaches for advancing phase retrieval and high-resolution image reconstruction in II, suggesting exciting future directions in the application of AI to optical astronomy.